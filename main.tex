\documentclass[11pt]{scrartcl}
\usepackage[utf8]{inputenc}
\usepackage[english]{babel}
\usepackage{hyperref}
\usepackage{graphicx}
\usepackage{listings}
\usepackage{xcolor}

% COLORS
% Use these definitions to change the color in the listings
\definecolor{green}{rgb}{0,0.6,0}

% CODE LISTINGS
\lstset{
    language=SQL,
    basicstyle=\footnotesize,
    commentstyle=\color{green},
    keywordstyle=\color{blue},
    numbers=left,
    numberstyle=\tiny\color{gray}
}

% biibliography 
% for citation styles see here: 
% https://www.overleaf.com/learn/latex/Natbib%20citation%20styles
%
% You can export references as bibtex (i.e. on google scholar) and copy them to the bibliography.bib
% The most important citation commands are: 
%   \citep{ref} for parenthesis
%   \citet{ref} for a text citation
\usepackage{natbib}
\bibliographystyle{abbrvnat}
\setcitestyle{authoryear,open={((},close={))}}


% CHANGE TO YOU NEEDS HERE
\title{HOBO Report: Your title goes here}
\author{Your Name}
\date{January 2021}

\begin{document}

\maketitle

\begin{abstract}
    An abstract is absolutely not necessary and you can simply drop or comment this part.
    Alternatively, you can use it to place your contact details:\medskip\par 
    \begin{tabular}{lc}
       student-id  & 1234567 \\
        e-mail & mail@example.com
    \end{tabular}
\end{abstract}

\section{Temperature indices}

An overview of temperature indices was implemented using a \texttt{VIEW} as defined in lising \ref{lst:view_1}

\lstinputlisting[label={lst:view_1},caption={create statement for view},]{./sql/temperature_indices.sql}
  

\section{Chapter 2:}


\bibliography{bibliography}
\end{document}